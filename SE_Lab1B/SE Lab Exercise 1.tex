\documentclass[11pt]{article}
\usepackage[margin=50pt]{geometry}
\usepackage{graphicx}
\graphicspath{/home/sibusiso/Desktop/.Login Page.PNG}
%Gummi|065|=)
\title{\textbf{SOFTWARE ENGINEERING LAB \\EXERCISE NO 1B\\ SOFTWARE REQUIREMENT SPECIFICATIONS}}
\author{\Large Sibusiso Mgidi 1141573\\
		\Large Ndivho Mamathuba 1053903\\
		\Large Culu Muzikawufani 1105477\\	
		\Large Siala Mbofholowo 1367404\\
		}

\date{\textbf{\Large 31 August 2018}}
\begin{document}
\linespread{1.5}
\maketitle{\textbf{\Large Fast Food on Wheels Delivery System}}
\tableofcontents
\newpage
\section{\Large Introduction}
\Large A client is requesting the implementation of custom tailored fast food delivery software management
system to run on a server (or on cloud computing service). The kinds of fast food
normally to be served include, Pizza, Hamburgers, Hot-dogs, etc. You can safely assume that
the restaurant has preset limited types of fast foods, except that clients can choose, add or refuse, condiments and dressings.
The system runs a back-end service which manages all the essential databases, e.g., available
dishes, customers and customer orders, stocks of ingredients for the dishes, employees -
part-timers and full-timers, managers, suppliers of stocks, etc. The front-end services feature
interfaces for interacting with the management systems.

\subsection{\Large Problem description of what is to be build}
The lack of the system that allows the custumer to make multiple orders and schedule delivery of all the orders in real time.
\subsection{\Large Methodology}
The goal of the project is to create a system that will allow the customer to make multiple orders online using possible devices such as tablet, laptop, mobile-devices and desktop application. The system will be used by the customer, restaurant manager and the system admininstrator.
\subsection{\Large Hypothesis}
The system should allow the customer to see the menu and add an item or multiple items to the menu, order the food, choose whether they want their order to be delivered or he/she will pick it up.If they want their order to be delivered then they should be able to trace track the order in real time.The user should also be able to choose which method of payment they will use between cash or paypal lastly the customer should be able to cancel the order.
\subsection{\Large Scope, keywords and proposed architecture}
The purpose of this analysis is to demonstrate the extent to which high-level systems concept and UML notation can be used to describe the functionality of this system. This study lays out a framework for a new system to be developed.\\
\textbf {\Large Proposed Architecture}\\
A System Sequence Diagram presents sequences for specific use cases and describe interactions between the user and the system in terms of an exchange of messages over time. It shows the details of events that are generated by actors from outside the  The system and gives detail on how operation are carried out. 

\includegraphics[width = 1.1\textwidth]{diagram.png}
\textbf {Keywords :}Customer, manager, administrator, data dictinary and software requirements specification\\

\section {\Large System's features}
\textbf {Register / sign in-}\\
Yes  we have the sign in or register feature. Email address , password and username are needed for registering. When it comes to registering , the customer must create a password of  6 character minimum but 9 characters maximum , the password must contain  at least one Upper case , one low case character , one symbol and one number and also the username should be a minimum of four characters but maximum of 10 characters   but the email address is going to be used as the primary key for all the users. When signing in , only the username and password would be needed. The delivery man is going to be added form the admin panel.

\textbf {Profile :}\\
Both the customer and delivery man are going to have profiles.

\textbf {Menu :}\\
The customer can view the all the menus and pick anything that they want from the Menu , The sum is going to be added together  when they  done picking.

\textbf {Pickup :} \\ 
The customer will have to be specify the drop off location then The delivery man is allowed to view the pickup location of the customer.

\textbf {Request ride}\\
The customer has to enter the destination where he/ she needs the drip off to be at .The Delivery man can either accept or reject the request ride  depending on  how far are they from the pickup location .The Delivery man must come to the customer.

\textbf{Push Notifications}\\
Both the customer and the delivery man are going to push notifications to each other , for example the delivery man can notify the customer when he has arrived at the drop location and The customer also notify the delivery man to cinfirm if he has arrived at the right location.

\textbf{Payment}\\
The customer can either choose to pay cash or digital .If the cash option is chosen , the delivery man will accept the cash where both customer and delivery man will sign that the payment went through by using  a payment book. If digital payment is chosen , then the delivery man can withdraw the amount to his bank account.

\textbf{Edit Profile}\\
The customer  can edit his/her profile anytime .For the delivery man , the editions of his profile is going to be done through the admin panel only.

\textbf{Cancel Ride}\\
No cancellation of the ride is allowed.

\textbf{Logout}
Both customer and delivery man clog out with just one tap




\section {\Large Overall Description}
\subsection{\Large Responsibility of each student in your group}
\begin{enumerate}
 \item Sibusiso Mgidi - Front End and Back End
 \item Culu Muzikawufani - Front End
 \item Siala Mbofholowo - Back End
 \item Ndivho Mamathuba - Back End  
\end{enumerate} 

\subsection{\Large Back End Tasks }
The system runs a back-end service which manages all the essential databases, e.g., available dishes, customers and customer orders, stocks of ingredients for the dishes, employees part-timers and full-timers, managers, suppliers of stocks, etc.
\subsection{\Large Front End Tasks }
The front-The front-end services feature interfaces for interacting with the mend developer is responsible for implementing visual elements that users see and interact with in a software application.
\section{\Large Identify actors, use cases and constrains}
\begin{itemize}
 \item Customer\\
 This is the principle customer who will ordeer food and make payment.
 \item Restaurant Manager\\
 This actor will hold the right to change the menu and enter the system to make any changes.
 \item System Administrator\\
 This actor is responsible for 
\end{itemize} 
\maketitle{\Large Customer Use Case}
\subsection{\Large Customer Registration}
\begin {itemize}
 \item  Username.
 \item Contact Details 
 \item Email address
 \item Phone numbers 
 \item Password 
\end{itemize}
\subsection{\Large Customer Possible Constrains}
The password must contain an uppercase letter and be a minimum of 8 characters
\\
 
\textbf{\Large User logins}
\begin{itemize}
 \item Username
 \item Password
\end{itemize}

\subsection{\Large Ordering use case}
\begin{itemize}
 \item Navigate the restaurant.
 \item Select an item from the menu.
 \item Add an item to their current order.
 \item Review their current order
 \item Remove an item/remove all items from their current order.
 \item Provide payment details.
 \item Receive confirmation in the form of an order number.
 \item View order placed.
 \item Track the order placed in real time.
\end{itemize} 

\subsection{\Large The restaurant manager interacts with the system to manage:}

\begin {itemize}
 \item Stock Inventory 
 \item Orders
 \item Customer database 
 \item Track delivery
 \item Accounting/Finacial Administration 
 \item Duties, off-days, leaves, etc.
 \item Generation of reports- Analytics
\end{itemize} 

\subsection{\Large System Administrator responsibilities:}
\begin {itemize}
  \item Setting up databases 
  \item Stopping/restarting system services
  \item General system maintanance
  \item Managing users accounts
  \item Performing backups
  \item Upgrading software
\end{itemize}

\section{\Large Data Dictionary} 
\subsection{\Large Customer Data Dictionary} 
Scope : This table contains demographic data of the applicant/customer. The table stores the attributes of the objects belonging to classes named CUSTOMER.\\
\textbf{\Large Attribute Listing : }\\
 
\textbf {customer id:}\\
Description :  Customer unique identification number generated by the system on an incremental basis\\
Type : 5 digit integer \\
Format : ***** \\
Note : This is the primary key, non-numeric values\\

\textbf {customer username:}\\
Description :  Customer username name\\
Type : 30 string character \\
Format :N/A \\
Note : Contains non-numeric and special characters. No numeric values\\

\textbf {customer email address:}\\
Description : Customer electronic mail address.\\
Type : 30 string character. \\
Format :Variable Character. \\
Note : The email will be verified online.\\

\textbf {customer password:}\\
Description : This is the customer password.\\
Type : 30 combination of strings and numerics. \\
Format :Variable Character. \\
Note : Take the combination of the strings, numerics and special characters\\

\textbf {customer phone number:}\\
Description : This is the customer phone number.\\
Type :  10 digit integer. \\
Format : Numerics\\
Note : Take the combination of the strings, numerics and special characters

\subsection{\Large Restaurant Manager Data Dictionary} 
Scope : This table contains the restaurant manager username, password and unique id. The table stores the attributes of the objects belonging to classes named MANAGER.
\\

\textbf{\Large Attribute Listing :}\\

\textbf {manager username:}\\
Description : Manager first name\\
Type : 30 string character \\
Format :N/A \\
Note : Contains non-numeric and special characters. No numeric values\\

\textbf {manager password:}\\
Description : This is the manager password.\\
Type : 30 combination of strings and numerics. \\
Format :Variable Character. \\
Note : Take the combination of the strings, numerics and special characters\\

\textbf {manager id:}\\
Description : Manager unique identification number generated by the system on an incremental basis\\
Type : 5 digit integer \\
Format : ***** \\
Note : This is the primary key, non-numeric values

\subsection{\Large Systems Addministrator Data Dictionary} 
Scope : This table contains the system's administrator username, password and unique id. The table stores the attributes of the objects belonging to classes named SYSTEM ADMINISTRATOR.
\\
\textbf{\Large Attribute Listing :}\\

\textbf {administrator username:}\\
Description : administrator first name\\
Type : 30 string character \\
Format :N/A \\
Note : Contains non-numeric and special characters. No numeric values\\

\textbf {administratdor password:}\\
Description : This is the manager password.\\
Type : 30 combination of strings and numerics. \\
Format :Variable Character. \\
Note : Take the combination of the strings, numerics and special characters\\

\textbf {administrator id:}\\
Description : Administrator unique identification number generated by the system on an incremental basis\\
Type : 5 digit integer \\
Format : ***** \\
Note : This is the primary key, non-numeric values

\section{\Large Prototyping} 
Prototyping refers to an initial stage of a software release in which development evolution and product fixes many occur before a bigger release is initiated.\\

\textbf {\Large Page 1 - Log in}\\
This interface clearly reminds users that they need to enter their user name and password to log in to this application. In addition, under the user name and password, there are forgot password options for those who have forgotten the password. For those who use this application for the first time, they can register to use the application by clicking the "Register" option at the bottom of dthe page.
\includegraphics[width = 0.9\textwidth]{Login.PNG}

\textbf {\Large Page 2 - Registration Form}\\
This interface clearly allows sthe user to enter their demographic data namely username, phone numbers, email address, password and confirm it. After the user has clicked the button create he/she will receive and email to verify the authenticity of the email.

\includegraphics[width = 0.9\textwidth]{Registration.PNG}


\textbf {\Large Page 2 - Restaurant Menu Form}\\
This is the homepage of the application, which shows activities and
advertisements, automatically recommend restaurants and search icon and
so on.

\includegraphics[width = 1.1\textwidth]{Menu.PNG}

\end{document}
